\documentclass[letterpaper,10pt,onecolumn]{article}
\usepackage[spanish]{babel}
\usepackage[utf8]{inputenc}
\usepackage{amsfonts}
\usepackage{amsthm}
\usepackage{amsmath}
\usepackage{mathrsfs}

\usepackage{enumitem}
\usepackage[pdftex]{color,graphicx}
\usepackage{hyperref}
\usepackage{listings}
\usepackage{calligra}
\usepackage{url}
%\usepackage{algpseudocode} 
\DeclareMathAlphabet{\mathcalligra}{T1}{calligra}{m}{n}
\DeclareFontShape{T1}{calligra}{m}{n}{<->s*[2.2]callig15}{}
\newcommand{\scripty}[1]{\ensuremath{\mathcalligra{#1}}}
\lstloadlanguages{[5.2]Mathematica}
\setlength{\oddsidemargin}{0cm}
\setlength{\textwidth}{490pt}
\setlength{\topmargin}{-40pt}
\addtolength{\hoffset}{-0.3cm}
\addtolength{\textheight}{4cm}

\begin{document}
\begin{center}

\includegraphics[width=490pt]{header.png}\\[0.5cm]

\textsc{\LARGE Cosmolog\'ia Moderna}\\[0.1cm]

\large Jaime E. Forero Romero\\[0.5cm]

\end{center}

\large \noindent\textsc{Nombre del curso:}  Cosmolog\'ia Moderna%Aqui  
                                %nombre del curso 
  
\noindent\textsc{C\'odigo del curso:} FISI XXXX %Aqui el codigo del
                                %curso 

\noindent\textsc{Unidad acad\'emica:} Departamento de F\'isica

\noindent\textsc{Periodo acad\'emico:} 201620 %Aqui el periodo,
                                %p.ej. 201510 

\noindent\textsc{Horario:} Ma y Ju, 8:30 a 9:50 %Aqui el horario, %p.ej. Ma y Ju, 10:00 a 11:20 

\noindent\rule{\textwidth}{1pt}\\[-0.3cm]

\normalsize \noindent\textsc{Nombre profesor(a) principal:} Jaime
E. Forero Romero%Aqui nombre del profesor principal 

\noindent\textsc{Correo electr\'onico:}
\href{mailto:je.forero@uniandes.edu.co}{\nolinkurl{je.forero@uniandes.edu.co}}
%Cambie address por su direccion de correo uniandes 

%\noindent\textsc{Horario y lugar de atenci\'on:} Ma y Ju 10:00 a
%11:00 AM, Oficina Ip208 %Aqui su horario y lugar de atencion, p.ej. Vi,
                     %15:00 a 17:00, Oficina Ip102  
%\\[-0.1cm]

\noindent\textsc{Nombre profesor(a) complementario(a):} %Aqui nombre
                                %del profesor complementario si aplica 

\noindent\textsc{Correo electr\'onico:}
\href{mailto:@uniandes.edu.co}{\nolinkurl{@uniandes.edu.co}}

%Cambie address por direccion de correo uniandes del profesor
%complementario 

%\noindent\textsc{Horario y lugar de atenci\'on:} %Aqui horario y
%lugar de atencion del profesor complementario, p.ej. Vi, 15:00 a
%17:00, Oficina Ip102 
%\\[-0.1cm]
%Repetir esto en caso de varios profesores complementarios

%\noindent\textsc{Nombre monitor(a):} %Aqui nombre del monitor si aplica

%\noindent\textsc{Correo electr\'onico:}
%\href{mailto:address@uniandes.edu.co}{\nolinkurl{address@uniandes.edu.co}}
%%Cambie address por direccion de correo uniandes del monitor 

%\noindent\textsc{Horario y lugar de atenci\'on:} %Aqui horario y
%lugar de atencion del monitor, p.ej. Vi, 15:00 a 17:00, Oficina Ip102 

\noindent\rule{\textwidth}{1pt}\\[-0.1cm]

\newcounter{mysection}
\addtocounter{mysection}{1}

\noindent\textbf{\large \Roman{mysection} \quad Introducci\'on}\\[-0.2cm]

%Este espacio es para hacer una introduccion al curso, evidenciando la
%propuesta metodologica. Debe ser clara y precisa. 

\noindent\normalsize 
La cosmología contemporánea ataca problemas de física fundamental que
retan nuestras concepciones sobre materia, energía, espacio y
tiempo. 
El curso ofrece una perspectiva general sobre los fundamentos
teóricos, observacionales y computacionales de esta área del
conocimiento. 
Comenzamos con elementos de la  cosmología Einsteniana y
terminamos tratando los grandes descubrimientos  recientes en la
actualidad,  la era de oro de la cosmología. Nos concentramos  en los
desarrollos de los últimos cincuenta años, en donde se ha avanzado en
la cosmología temprana, o sea, en entender los primeros trescientos
mil años  del universo. Se enfatizan los aspectos observacionales y
fenomenológicos y  se desarrolla una familiaridad con ciertas
herramientas del cálculo. Se  enfatiza la claridad en los conceptos
físicos, y no los formalismos. Se  introducen conceptos avanzados en
la medida que sean necesarios para  entender y aplicar a los temas que
se cubren. Usaremos conceptos extraídos  de astronomía, mecánica
clásica, relatividad general, termodinámica, física  estadística,
física nuclear, partículas elementales y campos cuánticos.  


Se asume que los estudiantes de este curso ya han cursado Mec\'anica
Cu\'antica I (FISI-3010).
\\[0.1cm]

\stepcounter{mysection}
\noindent\textbf{\large \Roman{mysection} \quad Objetivos}\\[-0.2cm]

%En este espacio se debe precisar el ente visor del curso y el
%proposito ideal al finalizar el curso. 
\noindent\normalsize Los objetivos principales del curso son:

\begin{itemize}

        \item Presentar los fundamentos f\'isicos y matem\'aicos del
          modelo cosmol\'ogico est\'andar. \\[-0.6cm]
        \item Presentar las bases astron\'omicas de
          la cosmolog\'ia observacional moderna.\\[-0.6cm]
        \item Estudiar la motivaci\'on cient\'ifica para los
          experimentos de siguiente generaci\'on en cosmolog\'ia
          observacional. \\[-0.2cm]
\end{itemize}

\stepcounter{mysection}
\noindent\textbf{\large \Roman{mysection} \quad Competencias a
  desarrollar}\\[-0.2cm] 

%En este espacio se describen las habilidades que el estudiante desarrollara en el transcurso del curso.

\noindent\normalsize Al finalizar el curso, se espera que el
estudiante est\'e en capacidad de: 

\begin{itemize}
\item Utilizar el formalismo matem\'atico adecuado para inferir distancias y
  tiempos cosmol\'ogicos a partir de mediciones observacionales. \\[-0.6cm]
\item Identificar la importancia de diferentes tipos de observaciones
  astron\'omicas en la medici\'on de par\'ametros cosmol\'ogicos.
\\[-0.6cm] 
\item Comprender los planteamientos y m\'etodos utilizados publicaciones
  cient\'ificas en el \'area de cosmolog\'ia observacional moderna. \\[-0.2cm]  
\end{itemize}

\stepcounter{mysection}
\noindent\textbf{\large \Roman{mysection} \quad Contenido por
  semanas}\\[-0.2cm] 

%Se expone de forma ordenada toda la tematica a tratar del curso. Debe planearse para 15 semanas.























\noindent\normalsize\textbf{\textsc{Semana 1.}}
Introducci\'on. Fundamentos observacionales de la cosmolog\'ia. 
\\[-0.3cm]

\noindent\textbf{\textsc{Semana 2.}} 
Un universo en expansi\'on. Gravedad Newtoniana y Einsteniana.
\\[-0.3cm]  

\noindent\textbf{\textsc{Semana 3.}} 
Consecuencias de la expansi\'on del Universo. Mediciones de distancias
y tiempos cosmol\'ogicos.
\\[-0.3cm]  

\noindent\textbf{\textsc{Semana 4.}} 
Historia t\'ermica del Universo.
\\[-0.3cm]  

\noindent\textbf{\textsc{Semana 5.}}
\'Exitos y problemas del modelo est\'andar cosmol\'ogico. Motivaci\'on
para la Inflaci\'on.
\\[-0.3cm]

\noindent\textbf{\textsc{Semana 6.}} 
Inestabilidad gravitacional.
\\[-0.3cm]  

\noindent\textbf{\textsc{Semana 7.}} 
Descripci\'on de fluctuaciones de densidad. 
\\[-0.3cm] 

\noindent\textbf{\textsc{Semana 8.}} 
Crecimientos de fluctuaciones de densidad. 
\\[-0.3cm]  

\noindent\textbf{\textsc{Semana 9.}} 
Evoluci\'on no lineal de la formaci\'on de estructura. 
\\[-0.3cm] 

\noindent\textbf{\textsc{Semana 10.}}  
Simulaciones de la formaci\'on de estructura del Universo.
\\[-0.3cm] 

\noindent\textbf{\textsc{Semana 11.}}  
Bariogénesis. Asimetría materia-antimateria.
\\[-0.3cm] 

\noindent\textbf{\textsc{Semana 12.}}  
El universo inflacionario. La singularidad inicial.
\\[-0.3cm]  

\noindent\textbf{\textsc{Semana 13.}} 
Herramientas de la astronom\'ia extragal\'actica.
\\[-0.3cm]  

\noindent\textbf{\textsc{Semana 14.}} 
Galaxias como trazadores de la estructura a gran escala del Universo. 
\\[-0.3cm] 

\noindent\textbf{\textsc{Semana 15.}} 
Nuevos experimentos de cosmolog\'ia observacional.
\\[-0.1cm]  


\stepcounter{mysection}
\noindent\textbf{\large \Roman{mysection} \quad
  Metodolog\'ia}\\[-0.2cm] 

%Se describen las tecnicas y metodos para el desarrollo exitoso del curso.

\noindent\normalsize Todas las clases se har\'an con presentaciones en
el tablero para desarrollar los conceptos y plantear ejercicios de
aplicaci\'on.\\[0.1cm]

\stepcounter{mysection}
\noindent\textbf{\large \Roman{mysection} \quad Criterios de
  evaluaci\'on}\\[-0.2cm] 

El curso tendr\'a tres parciales ($20 \%$ cada uno), entrega
peri\'odica de ejercicios ($20\%$) y un ex\'amen final ($20\%$). \\[0.1cm]


\stepcounter{mysection}
\noindent\textbf{\large \Roman{mysection} \quad
  Bibliograf\'ia}\\[-0.2cm] 

%Indicar los libros y la documentacion guia.

\noindent\normalsize Bibliograf\'ia principal:

\begin{itemize}
\item P. Schneider \textit{Extragalactic Astronomy and
  Cosmology: An Introduction},
  2006. (Biblioteca General - 523.112) \url{http://link.springer.com.ezproxy.uniandes.edu.co:8080/book/10.1007%2F978-3-540-33175-9})\\[-0.6cm]
\item J. A. Peacock, \textit{Cosmological Physics}, 2002. (Biblioteca
  General,  523.1 P211)\\[-0.6cm]
\item S. Dodelson \textit{Modern Cosmology}, 2003. (Biblioteca
  General, 523.1 D522)\\[-0.2cm]
\end{itemize} 

\noindent\normalsize Bibliograf\'ia complementaria:

\begin{itemize}
\item E. W. Kolb and M. S. Turner \textit{The Early Universe},
  1990. (Biblioteca General  523.1 K541)\\[-0.6cm]
\item A. Liddle, \textit{An introduction to modern cosmology}, 2003.
  (Biblioteca General, 523.1 L322 2003. )\\[-0.2cm] 
\end{itemize}


\end{document}
